%---------------------------------------------------------------
\chapter{Úvod}
%---------------------------------------------------------------
\setcounter{page}{1}

%---------------------------------------------------------------
\section{Motivace}
%---------------------------------------------------------------
Samotný lidský mozek je vybaven komplexním herutistickým strojem, který je využit pro řešení obrovského množství optimalizačních problémů. I přesto je však výzkum heuristik poměrně novým vědním oborem. Který od objevení prvních formálních studií heuristik, mezi lety 1940 -- 1980, urazil dalekou cestu.\cite{sorensen2017-1}

Po dlouhou dobu mělo studium heuristik obtíže být bráno vážně. Převládal názor, že oproti algoritmům postrádají heuristiky jakousi analytickou čistotu, získanou akademickým výzkumem. Nicméně se dá těžko argumentovat proti úspěchu, který si za poslední řadu let heuristiky vydobyly na poli optimalizačních problémů.\cite{sorensen2017-2} K zajištění jejich aktuálnosti a využití jejich plného potenciálu je třeba pokračování jejich výzkumu.

%Ze studia heuristik se postupně začaly  přineslo se po roce 1980 . jsou většinou vytvářeny tak, že i přes kvalitní řešení nad jednou doménou kombinatorických optimalizačních problémů, můžou na jiné pokulhávat.


%Hyperheuristiky se snaží řešit výpočetně těžké problémy novým přístupem abstrakce. Namísto využívání algoritmů optimalizovaných nad určitou třídou instancí problémů samostatně, se pokouší o jejich spolupráci. Kombinováním a nastavování těchto heuristik různými způsoby má tendenci dosahovat daleko lepších výsledků v širokém spetru.\cite{lehrbaum2011}

%jak prozname jejich kvalitu? rozepsat se o tom trochu, ze je potreba 

%---------------------------------------------------------------
\section{Cíle této práce}
%---------------------------------------------------------------

dopsat cíle práce


%---------------------------------------------------------------
\section{Aktuální stav výzkumu hyperheuristik}
%---------------------------------------------------------------

dopsat aktuální stav výzkumu hyperheuristik
\cite{DRAKE2020405}
