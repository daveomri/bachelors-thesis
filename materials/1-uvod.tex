%---------------------------------------------------------------
\chapter{Úvod}
%---------------------------------------------------------------
\setcounter{page}{1}

%---------------------------------------------------------------
\section{Motivace}
%---------------------------------------------------------------
Každý z nás je vybaven komplexním orgánem využívající heuristik pro řešení obrovského množství optimalizačních problémů. I přesto je však výzkum heuristik poměrně novým vědním oborem. Který od představení prvních formálních studií heuristik, mezi lety 1940 -- 1980, urazil dalekou cestu.\cite{sorensen2017-1} 

Po dlouhou dobu mělo studium heuristik obtíže být bráno vážně. Převládal názor, že oproti algoritmům postrádají jakousi analytickou čistotu, zaslouženou akademickým výzkumem. Nicméně se dá těžko argumentovat proti jejich přínosu u řešení optimalizačních problémů.\cite{sorensen2017-2} Kterým potvrzují svou aktuálnost a užitečnost pro další zkoumání.

%Ze studia heuristik se postupně začaly  přineslo se po roce 1980 . jsou většinou vytvářeny tak, že i přes kvalitní řešení nad jednou doménou kombinatorických optimalizačních problémů, můžou na jiné pokulhávat.


%Hyperheuristiky se snaží řešit výpočetně těžké problémy novým přístupem abstrakce. Namísto využívání algoritmů optimalizovaných nad určitou třídou instancí problémů samostatně, se pokouší o jejich spolupráci. Kombinováním a nastavování těchto heuristik různými způsoby má tendenci dosahovat daleko lepších výsledků v širokém spetru.\cite{lehrbaum2011}

%jak prozname jejich kvalitu? rozepsat se o tom trochu, ze je potreba 

%---------------------------------------------------------------
\section{Cíle této práce}
%---------------------------------------------------------------

Hlavním cílem této práce je zhodnocení optimalizačního frameworku SEAGE z pohledu aktuálního stavu výzkumu v oblasti hyperheuristik. 

%Tento cíl se v práci štěpí na menší podcíle, které k němu postupně směřují. Je třeba zajistit ověřitelná a reprodukovatelná data pro kritické porovnání.

Toho je v této práci dosaženo využitím nejlepších hyperheuristik účastníků mezinároní výzvy CHeSC2011 a heuristikami ve frameworku SEAGE.

%Toho je v této práci dosaženo porovnáním heuristik frameworku SEAGE s nejlepšími hyperheuristikami účastníků mezinárodní výzvy CHeSC2011. 

Aby však bylo možné vyvozovat nějaké závěry, je zapotřebí dat. Především takových, u kterých je možná ověritelnost jejich reprodukovatelností. Tím je v tomto kontextu myšlena množina řešení optimalizačních problémů, získaných vybranými heuristikami.

Reprodukovatelnost je zajištěna vlastním rozšířením implementace frameworku HyFlex, který byl právě v rámci CHeSC2011 využit. Stránky výzvy nabízejí referenční řešení jednotlivých účastníků nad množinou použitých intancí problémů (SAT, TSP). V rámci této práce tyto data reprodukuji na svém systému a porovnáním s referenčními ověřuji jejich pravdivost. 

Získaná data se ale nedají sama o sobě využít k vyvozování závěrů. K tomu je zapotřebí jejich odhocnocení. A to takovým způsobem, aby se zajistilo objektivní zhodnocení informace obsažené v nich. Nakonec tak i objektivní porovnání mezi sebou.

Pro tyto účely jsem oba frameworky rozšířil o evaluátor heuristik podle získaných řešení optimalizačních úloh nad jednotlivými doménami problémů. Součástí tohoto evaluátoru je implementovaná nově představená metrika. Myšlenka za ní je velice jednoduchá. Namapovat výsledek heuristik  na interval [0,1]. Kde nula značí snadno dostupné řešení (náhodným, nebo hladovým prohledávání). A naopak jednička výsledek optimální.

Metrika je taktéž využita v implementaci nové hyperheuristiky ve frameworku SEAGE.

Nad získanými výstupy evaluátoru vykonávám sérii experimentů k porovnání heuristik mezi sebou. Získané výsledky prezentuji v kapitole Experimenty.

%---------------------------------------------------------------
\section{Aktuální stav výzkumu hyperheuristik}
%---------------------------------------------------------------

dopsat aktuální stav výzkumu hyperheuristik
\cite{DRAKE2020405}
