%---------------------------------------------------------------
\chapter{Úvod}
%---------------------------------------------------------------
\setcounter{page}{1}

%---------------------------------------------------------------
\section{Motivace}
%---------------------------------------------------------------

Lidé využívají heuristiky již od nepaměti. Samotný lidský mozek je vybaven komplexním herutistickým strojem, který je využit pro řešení obrovského množství optimalizačních problémů. I přesto je však jejich výzkum (taktéž metaheuristik a hyperheuristik) poměrně novým vědním oborem. Který od vzniku prvních konceptů metaheruristik, kolem roku 1980, urazil dalekou cestu.\cite{sorensen2017}

Metaheuristiky jsou většinou vytvářeny tak, že i přes kvalitní řešení nad jednou doménou kombinatorických optimalizačních problémů, můžou na jiné pokulhávat.

Hyperheuristiky se snaží řešit výpočetně těžké problémy novým přístupem abstrakce. Namísto využívání algoritmů optimalizovaných nad určitou třídou instancí problémů samostatně, se pokouší o jejich spolupráci. Kombinováním a nastavování těchto heuristik různými způsoby má tendenci dosahovat daleko lepších výsledků v širokém spetru.\cite{lehrbaum2011}

%---------------------------------------------------------------
\section{Cíle této práce}
%---------------------------------------------------------------

dopsat cíle práce


%---------------------------------------------------------------
\section{Aktuální stav výzkumu hyperheuristik}
%---------------------------------------------------------------

dopsat aktuální stav výzkumu hyperheuristik

