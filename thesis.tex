% Template for theses recommended at Czech Technical University, Faculty of Information Technology
% inspired by lipics-v2019 documentclass
% Questions and help: <ondrej.guth@fit.cvut.cz>
% 2021: work in progress

\documentclass[a4paper,czech,unicode,twoside]{book}[2019/12/20]

%%%%%%%%%%%%%%%%%%%%%%%%%%%%%%%%%%
% FILL IN THIS INFORMATION
%%%%%%%%%%%%%%%%%%%%%%%%%%%%%%%%%%
\newcommand{\thesistitle}{Evaluace frameworku SEAGE}
\newcommand{\thesistype}{Bakalářská práce}
\newcommand{\thesisauthor}{David Omrai}
\newcommand{\supervisor}{Ing. Mgr. Ladislava Smítková Janků, Ph.D}
\newcommand{\department}{Katedra teoretické informatiky}
\newcommand{\yearofdefence}{2021}
%%%%%%%%%%%%%%%%%%%%%%%%%%%%%%%%%%
% END FILL IN
%%%%%%%%%%%%%%%%%%%%%%%%%%%%%%%%%%

%%%%%%%%%%%%%%%%%%%%%%%%%%%%%%%%%%
% TEMPLATE SETTINGS
% no need to modify anything within this section
%%%%%%%%%%%%%%%%%%%%%%%%%%%%%%%%%%
\title{\thesistitle}
\author{\thesisauthor}
\RequirePackage{babel}[2020/07/13] %localization
\RequirePackage{iftex}[2020/03/06]
\ifxetex
    \RequirePackage{ellipsis}[2020/05/22] %ellipsis workaround for XeLaTeX
\else
    \RequirePackage[utf8]{inputenc}[2018/08/11] %this file encoding
    \RequirePackage{lmodern}[2009/10/30]
\fi
\RequirePackage[bottom=4cm,footskip=4em]{geometry}[2020/01/02] %page layout
\RequirePackage{setspace}[2011/12/19] %line spacing in title page
\RequirePackage{listings}[2020/03/24]
\RequirePackage{xcolor}[2016/05/11]
\RequirePackage{multicol}[2019/12/09]
\RequirePackage{titlesec}[2019/10/16]
\RequirePackage{mathtools}[2020/03/24]
\RequirePackage{amssymb}[2013/01/14]
\RequirePackage[pdfpagelayout=TwoPageRight]{hyperref}[2020-05-15] % optional package
\RequirePackage{pdfpages}[2020/01/28]
\RequirePackage{fancyhdr}[2019/01/31]
\RequirePackage[labelsep=space,singlelinecheck=false,font={up,small},labelfont={sf,bf}]{caption}[2020/05/30]
\RequirePackage{amsthm}[2020/05/29]
\RequirePackage{emptypage}[2010/05/30]

%%% modra
 \definecolor{decoration}{RGB}{0, 122, 195} %CTU blue
 \definecolor{heading}{RGB}{0, 122, 195}
 \definecolor{headbackgroundgray}{RGB}{199, 219, 241} %light blue
 \definecolor{backgroundgray}{RGB}{199, 219, 241} %CTU light blue
 \definecolor{headgray}{rgb}{0.50,0.50,0.51}
 \definecolor{enumgray}{RGB}{0, 122, 195} %CTU blue

\bibliographystyle{plainurl}% the mandatory bibstyle

\setcounter{secnumdepth}{4}% numbering sections; 4: subsubsection

% code listing settings
\renewcommand\lstlistingname{Výpis kódu}
\renewcommand\lstlistlistingname{Seznam výpisů kódu}
\lstset{basicstyle=\small\ttfamily,keywordstyle=\bfseries,%
        backgroundcolor=\color{backgroundgray},%
        frame=single,framerule=0pt,xleftmargin=\fboxsep,xrightmargin=\fboxsep}
% code listing setting end

% captions settings
\makeatletter
\@ifpackagelater{hyperref}{2009/12/09}
  {\captionsetup{compatibility=false}}%cf. http://groups.google.de/group/comp.text.tex/browse_thread/thread/db9310eb540fbbd8/42e30f3b7b3aa17a?lnk=raot
  {}
\makeatother
\DeclareCaptionLabelFormat{boxed}{%0.61,0.61,0.61
  \kern0.05em{\color{decoration}\rule{0.73em}{0.73em}}%
  \hspace*{0.67em}\bothIfFirst{#1}{~}#2}
\captionsetup{labelformat=boxed}
\captionsetup[table]{position=top}
% captions settings end

% lists settings
\setlength\leftmargini  \parindent
\setlength\leftmarginii {1.2em}
\setlength\leftmarginiii{1.2em}
\setlength\leftmarginiv {1.2em}
\setlength\leftmarginv  {1.2em}
\setlength\leftmarginvi {1.2em}
\renewcommand\labelenumi{%
  \textcolor{enumgray}{\headfont\bfseries\upshape\mathversion{bold}\theenumi.}}
\renewcommand\labelenumii{%
  \textcolor{enumgray}{\headfont\bfseries\upshape\mathversion{bold}\theenumii.}}
\renewcommand\labelenumiii{%
  \textcolor{enumgray}{\headfont\bfseries\upshape\mathversion{bold}\theenumiii.}}
\renewcommand\labelenumiv{%
  \textcolor{enumgray}{\headfont\bfseries\upshape\mathversion{bold}\theenumiv.}}
\makeatletter
\renewcommand\labelitemi{%
  \textcolor{enumgray}{\ifnum\@listdepth=\@ne
                                  \rule{0.67em}{0.33em}%
                                \else
                                  \rule{0.45em}{0.225em}%
                                \fi}}
\makeatother
\renewcommand\labelitemii{%
  \textcolor{enumgray}{\rule{0.45em}{0.225em}}}
\renewcommand\labelitemiii{%
  \textcolor{enumgray}{\headfont\bfseries\textasteriskcentered}}
\renewcommand\labelitemiv{%
  \textcolor{enumgray}{\headfont\bfseries\textperiodcentered}}
\renewcommand*\descriptionlabel[1]{%
  \hspace\labelsep\textcolor{enumgray}{\headfont\bfseries\mathversion{bold}#1}}
% lists settings end

\def\headfont{\rmfamily} %font for headings (chapter, section, etc.)

% frontmatter headings
\def\frontchapterfont{\Large \bfseries}
\def\frontsectionfont{\large\bfseries}
\def\frontsubsectionfont{\large}
\def\frontsubsubsectionfont{\bfseries}
% frontmatter headings end

% frontmatter pseudochapters: named part without printing actual chapter heading
\makeatletter
\newcommand{\frontchapternotprinted}[1]{%
  \begingroup
  \let\@makechapterhead\@gobble % make \@makechapterhead do nothing
  \let\cleardoublepage\clearpage
  \chapter{#1}
  \endgroup
}
\makeatother
% frontmatter pseudochapters end

% theorems, proofs, definitions, etc. end
\makeatletter
\thm@headfont{%
  \textcolor{decoration}{$\blacktriangleright$}\nobreakspace\headfont\bfseries}
\def\th@remark{%
  \thm@headfont{%
    \textcolor{decoration}{$\blacktriangleright$}\nobreakspace\headfont}%
  \normalfont % body font
  \thm@preskip\topsep \divide\thm@preskip\tw@
  \thm@postskip\thm@preskip
}
\def\@endtheorem{\endtrivlist}%\@endpefalse
\renewcommand\qedsymbol{\textcolor{decoration}{\ensuremath{\blacktriangleleft}}}
\renewenvironment{proof}[1][\proofname]{\par
  \pushQED{\qed}%
  \normalfont \topsep6\p@\@plus6\p@\relax
  \trivlist
  \item[\hskip\labelsep
        \color{black}\headfont\bfseries
    #1\@addpunct{.}]\ignorespaces
}{%
  \popQED\endtrivlist%\@endpefalse
}
\makeatother
\theoremstyle{plain}
\newtheorem{theorem}{Věta}
\newtheorem{lemma}[theorem]{Pomocné tvrzení}
\newtheorem{corollary}[theorem]{Důsledek}
\newtheorem{proposition}[theorem]{Návrh}
\newtheorem{definition}[theorem]{Definice}
\theoremstyle{definition}
\newtheorem{example}[theorem]{Příklad}
\theoremstyle{remark}
\newtheorem{note}[theorem]{Poznámka}
\newtheorem*{note*}{Poznámka}
\newtheorem{remark}[theorem]{Pozorování}
\newtheorem*{remark*}{Pozorování}
\numberwithin{theorem}{chapter}
% theorems, proofs, definitions, etc. end

% table of contents colored chapters
\makeatletter
\let\stdl@chapter\l@chapter
\renewcommand*{\l@chapter}[2]{%
  \stdl@chapter{\textcolor{heading}{#1}}{\textcolor{heading}{#2}}}
\makeatother
% table of contents colored chapters end

% headers and footers
\makeatletter
\def\ps@plain{%chapter beginning
    \let\@evenhead\@empty%
    \let\@oddhead\@empty%
    \def\@evenfoot{\bfseries\color{headgray}\hfill\thepage\hfill}%
    \def\@oddfoot{\bfseries\color{headgray}\hfill\thepage\hfill}}
\makeatother
\renewcommand{\chaptermark}[1]{\markboth{#1}{}}
\renewcommand{\sectionmark}[1]{\markright{#1}}
\pagestyle{fancy}
\fancyhf{}

\fancyhead[LE]{\leavevmode\smash{\llap{\color{headgray} \bfseries \hspace*{4em}}}}
\fancyhead[RE]{\color{headgray}\bfseries\nouppercase{\leftmark}}
\fancyhead[RO]{\leavevmode\smash{\rlap{\hspace*{4em}\color{headgray}\bfseries}}}
\fancyhead[LO]{\color{headgray}\bfseries\nouppercase{\rightmark}}

\fancyfoot[LE]{\bfseries\color{headgray}\hfill\thepage\hfill}
\fancyfoot[RE]{\bfseries\color{headgray}\hfill\thepage\hfill}
\fancyfoot[LO]{\bfseries\color{headgray}\hfill\thepage\hfill}
\fancyfoot[RO]{\bfseries\color{headgray}\hfill\thepage\hfill}
\renewcommand{\headrulewidth}{0pt}
% headers and footers end

% title page
\renewcommand{\maketitle}{\begin{titlepage}%
\newgeometry{left=67mm,top=80mm,right=40mm}%\thispagestyle{empty}%
\noindent{\large\headfont\noindent\thesistype}
\vskip 3mm
\noindent{\noindent\huge\headfont\bfseries\color{black}\begin{onehalfspace}\MakeUppercase{\thesistitle}\end{onehalfspace}}
\vskip 35mm
\noindent{\large \headfont \bfseries \thesisauthor}

\vfill

\noindent{\headfont Fakulta informačních technologií ČVUT v Praze\\
\department\\
Vedoucí: \supervisor\\
\today}\end{titlepage}
\restoregeometry
}
%title page end

\newenvironment{abstrakt}{%
  \vspace*{18mm}
  \noindent
  {{\frontchapterfont\begin{flushleft}{\color{heading}Abstrakt}\end{flushleft}}}%
  \bigskip
  \noindent\ignorespaces}

\newenvironment{abstract}{%
  \vskip\bigskipamount
  \noindent
  {{\frontchapterfont\begin{flushleft}{\color{heading}Abstract}\end{flushleft}}}%
  \bigskip
  \noindent\ignorespaces}
  
\newenvironment{prohlaseni}{
  {{\frontchapterfont\begin{flushright}{\color{heading}Prohlášení}\end{flushright}}}%
  \bigskip
  \noindent\ignorespaces}

%%%%%%%%%%%%%%%%%%%%%%%%%%%%%%%%%%
% TEMPLATE SETTINGS END
%%%%%%%%%%%%%%%%%%%%%%%%%%%%%%%%%%


%%%%%%%%%%%%%%%%%%%%%%
% DEMO CONTENTS SETTINGS
% You may choose to modify this part.
%%%%%%%%%%%%%%%%%%%%%%
\usepackage{dirtree}
\usepackage{lipsum,tikz}
%%%%%%%%%%%%%%%%%%%%%%
% DEMO CONTENTS SETTINGS END
%%%%%%%%%%%%%%%%%%%%%%

\begin{document}
\frontmatter
%%%%%%%%%%%%%%
% FRONTMATTER SETTINGS
% no need to modify this part
%%%%%%%%%%%%%%
\titleformat
{\chapter} % command
% [display] % shape
{} % format
{} % label
{} % sep
{\color{heading}\frontchapterfont \raggedleft} % before-code
[\vskip -2em] % after-code

\titleformat
{\section}
{\frontsectionfont\color{heading}}
{}
{}
{}
  
% \titleformat{\subsection}
%   {\frontsubsectionfont\color{heading}}{{{\color{black}\thesubsection}}}{1em}{}[\vskip -1em]
  
% \titleformat{\subsubsection}
%   {\frontsubsubsectionfont\color{heading}}{{{\color{black}\thesubsubsection}}}{1em}{}[\vskip -1em]

\makeatletter
\@openrightfalse
\makeatother
%%%%%%%%%%%%%%
% FRONTMATTER SETTINGS END
%%%%%%%%%%%%%%


\includepdf{assignment-include.pdf}

\thispagestyle{empty}
\cleardoublepage

\maketitle
% \clearpage

%%%%%%%%%%%%%%%%%%%%%%%%%%%%%%
% PREPRINT
% no need to modify
%%%%%%%%%%%%%%%%%%%%%%%%%%%%%%

	\clearpage
	\thispagestyle{empty}
	

		~
	\vfill
	
	{\small
			\noindent České vysoké učení technické v Praze \\
		\noindent Fakulta informačních technologií \\
	\noindent \textcopyright{} 2020 \thesisauthor. Všechna práva vyhrazena.\\
		\noindent \textit{Tato práce vznikla jako školní díla na Českém vysokém učení technické v Praze, Fakultě informačních technologií. Práce je chráněna právními předpisy a mezinárodními úmluvami o právu autorském a právech souvisejících s právem autorským. K jejímu užití, s výjimkou bez uplatněných zákonných licencí nad rámec oprávnění uvedených v Prohlášení je nezbytný souhlas autora.}
	
	\vspace{1em}
	
	\noindent Odkaz na tuto práci: \thesisauthor. \textit{\thesistitle}. \thesistype. České vysoké učení technické v Praze, Fakulta informačních technologií, \yearofdefence.
}
%%%%%%%%%%%%%%%%%%%%%%%%%%%%%%
% PREPRINT END
%%%%%%%%%%%%%%%%%%%%%%%%%%%%%%

\tableofcontents
\listoffigures
\begingroup
\let\clearpage\relax
\listoftables
\lstlistoflistings
\endgroup

%%%%%%%%%%%%%%%%%%%
% ACKNOWLEDGMENT
% FILL IN / MODIFY
%%%%%%%%%%%%%%%%%%%
\frontchapternotprinted{Poděkování}
~
\vfill
\hskip 0cm \begin{minipage}{0.7\textwidth}
\textit{Chtěl bych poděkovat především Ing. Mgr. Ladislavě Smítků Jánků, Ph.D za odborné vedení, trpělivost a ochotu, kterou mi v průběhu tvorby této bakalářské práce věnovala. Dále bych rád poděkoval Ing. Richardu Málkovi, za jeho nesčetné rady během spolupráce na vývoji frameworku SEAGE. Na závěr chci poděkovat i své rodině, která mi byla oporou během zpracovávání této práce.}
\end{minipage}

\vfill

\vfill
%%%%%%%%%%%%%%%%%%%
% ACKNOWLEDGMENT END
%%%%%%%%%%%%%%%%%%%



%%%%%%%%%%%%%%%%%%%
% ACKNOWLEDGMENT
% FILL IN / MODIFY
%%%%%%%%%%%%%%%%%%%
\frontchapternotprinted{Prohlášení}
~
\vfill
% INSTRUCTIONS
% ENG: choose one of approved texts of the declaration. DO NOT CREATE YOUR OWN. Find the approved texts at https://courses.fit.cvut.cz/SFE/download/index.html#_documents (document Declaration for FT in English)
% CZE/SLO: Vyberte jedno z fakultou schvalenych prohlaseni. NEVKLADEJTE VLASTNI TEXT. Schvalena prohlaseni najdete zde: https://courses.fit.cvut.cz/SZZ/dokumenty/index.html#_dokumenty (prohlášení do ZP)
\begin{prohlaseni}
Prohlašuji, že jsem předloženou práci vypracoval samostatně a že jsem uvedl veškeré
použité informační zdroje v souladu s Metodickým pokynem o dodržování etických
principů při přípravě vysokoškolských závěrečných prací.
\newline
\newline
Beru na vědomí, že se na moji práci vztahují práva a povinnosti vyplývající ze zákona
č. 121/2000 Sb., autorského zákona, ve znění pozdějších předpisů. V souladu s ust.
§ 2373 odst. 2 zákona č. 89/2012 Sb., občanský zákoník, ve znění pozdějších předpisů,
tímto uděluji nevýhradní oprávnění (licenci) k užití této mojí práce, a to včetně všech 
počítačových programů, jež jsou její součástí či přílohou a veškeré jejich
dokumentace (dále souhrnně jen „Dílo“), a to všem osobám, které si přejí Dílo užít.
Tyto osoby jsou oprávněny Dílo užít jakýmkoli způsobem, který nesnižuje hodnotu
Díla a za jakýmkoli účelem (včetně užití k výdělečným účelům). Toto oprávnění je
časově, teritoriálně i množstevně neomezené. Každá osoba, která využije výše
uvedenou licenci, se však zavazuje udělit ke každému dílu, které vznikne (byť jen
zčásti) na základě Díla, úpravou Díla, spojením Díla s jiným dílem, zařazením Díla
do díla souborného či zpracováním Díla (včetně překladu) licenci alespoň ve výše
uvedeném rozsahu a zároveň zpřístupnit zdrojový kód takového díla alespoň
srovnatelným způsobem a ve srovnatelném rozsahu, jako je zpřístupněn zdrojový
kód Díla.

\vskip 1cm
\noindent
V Praze dne 9.\;května 2021 \hspace{.3\textwidth} \dotfill
\end{prohlaseni}
%%%%%%%%%%%%%%%%%%%
% DECLARATION END
%%%%%%%%%%%%%%%%%%%


%%%%%%%%%%%%%%%%%%%
% ABSTRACT
% FILL IN / MODIFY
%%%%%%%%%%%%%%%%%%%
\frontchapternotprinted{Abstrakt}
\begin{abstrakt}% Enter abstract in CZECH.

Tato práce je zaměřena na zhodnocení optimalizačního frameworku SEAGE z pohledu aktuálního stavu výzkumu v oblasti hyperheuristik. Pro toto zhodnocení práce představuje novou metriku, která algoritmy ohodnocuje objektivně dle kvality jejich řešení instancí problémů. 

Porovnání je provedeno mezi již implementovanými heuristikami, nově představenou hyperheuristikou v frameworku SEAGE a hyperheuristikami účastníků mezinárodní výzvy CHeSC2011. V rámci této výzvy měli účastníci za úkol implementaci hyperheuristik ve frameworku HyFlex. Ten je využit především k získání řešení těchto účastníků nad různými instancemi problémů. Získaná data se společně s těmi z frameworku SEAGE ohodnotili novou metrikou a v této práci vyvodily závěry.

Hlavním cílem této práce je návrh a implementace propojení nově představeného evaluátoru algoritmů s heuristikami z frameworků SEAGE a HyFlex. Demonstrace tvorby hyperheuristik a kritické zhodnocení optimalizačního frameworku SEAGE z pohledu aktuálnosti stavu výzkumu hyperheuristik.
\end{abstrakt}

\vskip 0.5cm

{\noindent\color{heading}\bfseries\headfont Klíčová slova\hspace{1em}}{SAT, TSP, metaheuristika, hyperheuristika, implementace hyperheuristiky, ohodnocení výsledků heuristik, interpretace řešení, SEAGE, HyFlex, CHeSC2011, ISEA, AdapHH, EPH, PHUNTER}

\vskip 1cm

\begin{abstract}% Enter abstract in ENGLISH.
This work
\end{abstract}

\vskip 0.5cm

{\noindent\color{heading}\bfseries\headfont Keywords\hspace{1em}}{metaheuristic, hyperheuristic, SAT, TSP, CHeSC2011, ACO, GA, ISEA, AdapHH, EPH, SEAGE, HyFlex}
%%%%%%%%%%%%%%%%%%%
% ABSTRACT END
%%%%%%%%%%%%%%%%%%%

%%%%%%%%%%%%%%%%%%%
% SUMMARY
% FILL IN / MODIFY
% OR REMOVE ENTIRELY (upon agreement with your supervisor)
% (apropriate to remove in most theses)
%%%%%%%%%%%%%%%%%%%
\chapter{Shrnutí}
\setlength{\columnsep}{1cm}
\begin{multicols}{2}
{\small

\section*{Motivace}

dopsat motivaci

\section*{Cíl práce}

dopsat cíle práce

\section*{Postup}

dopsat postup

\section*{Výsledky práce}

dopsat výsledky

\section*{Závěr}

dopsat závěr
}
\end{multicols}
%%%%%%%%%%%%%%%%%%%
% SUMMARY END
%%%%%%%%%%%%%%%%%%%

%%%%%%%%%%%%%%%%%%%
% ABBREVIATIONS
% FILL IN / MODIFY
% OR REMOVE ENTIRELY
%%%%%%%%%%%%%%%%%%%
\chapter{Seznam zkratek}
	
\begin{tabular}{rl}
SEAGE & A hyper-heuristic framework for collaboration of metaheuristic algorithms\\
HyFlex & A Benchmark Framework for Cross-domain Heuristic Search\\
CHeSC2011 & Cross-Domain Heuristic Search Challange 2011\\
TSP & Travelling Salesman Problem\\
SAT & Boolean satisfiability problem\\
GA & Genetic Algorithm\\
ACO & Ant Colony Optimalization\\
PHUNTER & Pearl Hunter\\
ISEA & Iterated Search Driven by Evolutionary Algorithm\\
AdapHH & An adaptive hyper-heuristic\\
EPH & Evolutionary Programming Hyper-heuristic\\ 
\end{tabular}
%%%%%%%%%%%%%%%%%%%
% ABBREVIATIONS END
%%%%%%%%%%%%%%%%%%%

\mainmatter

%%%%%%%%%%%%%%%%%%%%
% MAINMATTER SETTINGS
% no need to modify this part
%%%%%%%%%%%%%%%%%%%%
\makeatletter
\@openrighttrue
\makeatother

\titleformat
{\chapter} % command
[display] % shape
{\headfont \LARGE \bfseries \raggedleft} % format
{ \textcolor{headbackgroundgray}{{\titlerule*[1pc]{\rule{0.25em}{0.25em}}}} \hspace{0.5ex} \color{black} Kapitola \thechapter } % label
{-0.3cm} % sep
{\color{heading} \Huge \vskip 0.5cm} % before-code
[\vskip 1cm] % after-code

%vzhled nadpisů sekcí
\titleformat{\section}
  {\headfont\Large\bfseries\color{heading}}{\colorbox{headbackgroundgray}{{\color{black}\thesection}}}{1em}{}%[\vskip -1em]
  
\titleformat{\subsection}
  {\headfont\Large\bfseries\color{heading}}{{{\color{black}\thesubsection}}}{1em}{}%[\vskip -1em]
  
\titleformat{\subsubsection}
  {\headfont\large\bfseries\color{heading}}{{{\color{black}\thesubsubsection}}}{1em}{}%[\vskip -1em]

%%%%%%%%%%%%%%%%%%%%
% MAINMATTER SETTINGS END
%%%%%%%%%%%%%%%%%%%%

%%%%%%%%%%%%%%%%%%%
% THE THESIS
% MODIFY ANYTHING BELOW THIS LINE
%%%%%%%%%%%%%%%%%%%

%---------------------------------------------------------------
\chapter{Úvod}
%---------------------------------------------------------------
\setcounter{page}{1}

%---------------------------------------------------------------
\section{Motivace}
%---------------------------------------------------------------

Lidé využívají heuristiky již od nepaměti. Samotný lidský mozek je vybaven komplexním herutistickým strojem, který je využit pro řešení obrovského množství optimalizačních problémů. I přesto je však jejich výzkum (taktéž metaheuristik a hyperheuristik) poměrně novým vědním oborem. Který od vzniku prvních konceptů metaheruristik, kolem roku 1980, urazil dalekou cestu.\cite{sorensen2017}

Metaheuristiky jsou většinou vytvářeny tak, že i přes kvalitní řešení nad jednou doménou kombinatorických optimalizačních problémů, můžou na jiné pokulhávat.

Hyperheuristiky se snaží řešit výpočetně těžké problémy novým přístupem abstrakce. Namísto využívání algoritmů optimalizovaných nad určitou třídou instancí problémů samostatně, se pokouší o jejich spolupráci. Kombinováním a nastavování těchto heuristik různými způsoby má tendenci dosahovat daleko lepších výsledků v širokém spetru.\cite{lehrbaum2011}

%---------------------------------------------------------------
\section{Cíle této práce}
%---------------------------------------------------------------

dopsat cíle práce


%---------------------------------------------------------------
\section{Aktuální stav výzkumu hyperheuristik}
%---------------------------------------------------------------

dopsat aktuální stav výzkumu hyperheuristik


%---------------------------------------------------------------
\chapter{Teorie}
%---------------------------------------------------------------

%---------------------------------------------------------------
\section{Optimalizační problémy}
%---------------------------------------------------------------

napsat druhy a typy optimalizačních problémů, o co jde

%---------------------------------------------------------------
\subsection{Problém splnitelnosti booleovské formule}
%---------------------------------------------------------------

přiblížit čím se zabývá


%---------------------------------------------------------------
\subsection{Problém obchodního cestujícího}
%---------------------------------------------------------------

přiblížit čím se zabývá


%---------------------------------------------------------------
\section{Heuristiky}
%---------------------------------------------------------------
neco o nich 

%---------------------------------------------------------------
\subsection{Náhodné řešení}
%---------------------------------------------------------------
neco o náhodném řešení

%---------------------------------------------------------------
\subsection{Hladový algoritmus}
%---------------------------------------------------------------
neco o náhodném hladovém algoritmu

%---------------------------------------------------------------
\section{Metaheuristiky}
%---------------------------------------------------------------
neco o metaheuristikach kde se daji vyuzit atd..

%---------------------------------------------------------------
\subsection{Klasifikace}
%---------------------------------------------------------------
lokalni a globalni prohledavani

%---------------------------------------------------------------
\subsection{Genetický algoritmus}
%---------------------------------------------------------------
neco o genetice

dopsat teorii o metahuristice

%---------------------------------------------------------------
\subsection{Tabu prohledávání}
%---------------------------------------------------------------
neco o tabu prohledávání

%---------------------------------------------------------------
\subsection{Simulované žíhání}
%---------------------------------------------------------------
neco o zihani

%---------------------------------------------------------------
\subsection{Mravenčí kolonie}
%---------------------------------------------------------------
neco o mravensi kolonii


%---------------------------------------------------------------
\section{Hyperheuristiky}
%---------------------------------------------------------------

dopsat teorii o hyperheuristice

%---------------------------------------------------------------
\subsection{Klasifikace}
%---------------------------------------------------------------
lokalni a globalni vyhledavani


%---------------------------------------------------------------
\subsection{EPH}
%---------------------------------------------------------------
neco o eph

%---------------------------------------------------------------
\subsection{GIHH/AdapHH}
%---------------------------------------------------------------
neco o GIHH

%---------------------------------------------------------------
\subsection{ISEA}
%---------------------------------------------------------------
neco o ISEA

%---------------------------------------------------------------
\subsection{LeanGIHH}
%---------------------------------------------------------------
neco o leangihh\cite{adriaensen2016case}

%---------------------------------------------------------------
\subsection{PHUNTER}
%---------------------------------------------------------------
neco o pearl hunter


%---------------------------------------------------------------
\section{Optimalizační frameworky}
%---------------------------------------------------------------

%---------------------------------------------------------------
\subsection{SEAGE}
%---------------------------------------------------------------
neco o seage

%---------------------------------------------------------------
\subsection{HyFlex}
%---------------------------------------------------------------
neco o hyflexu



%---------------------------------------------------------------
\section{Metriky}
%---------------------------------------------------------------
neco o metrice

%---------------------------------------------------------------
\subsection{Bodový systém formule 1}
%---------------------------------------------------------------
neco o bodovem systemu pouzitem v hyflexu

%---------------------------------------------------------------
\subsection{UnitMetric}
%---------------------------------------------------------------
neco o nove metrice













%---------------------------------------------------------------
\chapter{Implementace}
%---------------------------------------------------------------

%---------------------------------------------------------------
\section{Prvotní představa implementace evaluátoru k porovnání heuristik}
%---------------------------------------------------------------
napsat o tom, jak se chtela vyuzit metrika f1 a jeji problemy
co bylo prvne zamysleno, ukazat predstavovane napojeni metriku a proc
to nemohlo fungovat, protoze maji schovane zdrojove kody a jak poskytuji heuristikam pristup k reseni

ze neni sance mit instanci problemu, ze hh nevi o problemu atd..

%---------------------------------------------------------------
\section{Finální implementace evaluátoru k porovnání heuristik}
%---------------------------------------------------------------
co se muselo udelat a s cim se skoncilo
novy srozumitelny zpusob srovnani

%---------------------------------------------------------------
\subsection{Představení nového evaluátoru heuristik}
%---------------------------------------------------------------
co se muselo udelat a s cim se skoncilo

%---------------------------------------------------------------
\section{Evaluátor ve frameworku HyFlex}
%---------------------------------------------------------------
co se muselo udelat a s cim se skoncilo
nejenom nova metrika, ale rekonstrukce souteze, reprodukce vysledku, prepsani trid a 
upraveni ukladani dat
%---------------------------------------------------------------
\subsection{Použité třídy}
%---------------------------------------------------------------


%---------------------------------------------------------------
\section{Evaluátor ve frameworku SEAGE}
%---------------------------------------------------------------
co se implementovalo

%---------------------------------------------------------------
\subsection{Použité třídy}
%---------------------------------------------------------------

%---------------------------------------------------------------
\section{Implementace heuristiky ve frameworku SEAGE}
%---------------------------------------------------------------
ukazat jak jsem vyuzil pripravenou metriku, co z toho vypadlo
%---------------------------------------------------------------
\chapter{Experimenty}
%---------------------------------------------------------------

%---------------------------------------------------------------
\section{Experiment}
%---------------------------------------------------------------

doplnit experimenty
%---------------------------------------------------------------
\chapter{Závěr}
%---------------------------------------------------------------

nejaka omacka na konec

\appendix

\include{appendix}

\backmatter

\bibliography{bib-database}

\include{medium}

\end{document}
